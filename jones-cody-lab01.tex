\documentclass[12pt]{article}
\usepackage[margin=2.5cm]{geometry}
\usepackage{amsmath}
\usepackage{graphicx}
\usepackage{circuitikz}

\title{Lab 01: Passive Circuits}
\author{Cody Jones}
\date{September 14, 2018}

\begin{document}

\maketitle

\section{Introduction}
In steady state circuit analysis, it is often convenient to replace real currents and voltages by complex quantities called \textit{phasors}.
So we might represent a sinusoidal current and voltage as follows:

$$I(t) \rightarrow \hat{I}(\omega)e^{i\omega t}$$,
$$V(t) \rightarrow \hat{V}(\omega)e^{i\omega t}$$.

If we restrict ourselves to passive, linear circuit elements, there is always a linear relationship between the voltage across the terminals of the element and the current that flows through it.
Our way of representing physical currents and voltages in the language of complex numbers motivates the definition of a complex quantity analogous to resistance:

\begin{equation}
Z(\omega) = \frac{\hat{V}(\omega)}{\hat{I}(\omega)},
\end{equation}

\noindent where $Z(\omega)$ is called the \textit{impedance} and has units of Ohms.

In addition to




\begin{figure}[h]
\centering
\includegraphics{transfer-function.png}
\end{figure}

\begin{center}

\begin{circuitikz}[scale=4]
\draw (0,0) to[american voltage source, l_=$V_{in}$] (0,1.5)
to[R, l_=$R_s$] (0.75,1.5)
to[diode] (1.5,1.5)
to[R, l_=$r$, -*] (2.25,1.5)
to[C, l_=$C$, -*] (2.25,0)
-- (0,0);
\draw (2.25,1.5) -- (3,1.5)
to[R, l_=$R$] (3,0)
-- (2.25,0);
\end{circuitikz}
\end{center}

\end{document}
